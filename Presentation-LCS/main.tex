\documentclass{beamer}
\usepackage{subfiles}
\usepackage{tikz}
\usepackage{color}
\usepackage[]{algorithm2e}
\usepackage[utf8]{inputenc}


% There are many different themes available for Beamer. A comprehensive
% list with examples is given here:
% http://deic.uab.es/~iblanes/beamer_gallery/index_by_theme.html
%\usetheme{Darmstadt}
%\usetheme{default}
%\usetheme{Ilmenau}
%\usetheme{JuanLesPins}
%\usetheme{Luebeck}
\usetheme{Madrid}
%\usetheme{Malmoe}
%\usetheme{Marburg}
%\usetheme{Montpellier}
%\usetheme{PaloAlto}
%\usetheme{Pittsburgh}
%\usetheme{Rochester}
%\usetheme{Singapore}
%\usetheme{Szeged}
%\usetheme{Warsaw}

\title{Longest Common Subsequence}

\author{Tanveer Muttaqueen \and Subangkar Karmaker Shanto}

\institute[Universities of Somewhere and Elsewhere] % (optional, but mostly needed)
{
Department of Computer Science \& Engineering\\
Bangladesh University of Engineering and Technology
}

\date{July, 2018}

\begin{document}
\begin{frame}
  \titlepage
\end{frame}


\begin{frame}{Problem Statement}
\setbeamercovered{dynamic}
\begin{enumerate}
    \onslide\item<1-> \textbf{Input:} Two strings $s$ and $t$. 
    \onslide\item<2-> \textbf{Output:} longest common subsequence of $s$ and $t$.
\end{enumerate}
\end{frame}

\begin{frame}{Subsequence}
\setbeamercovered{dynamic}
\begin{enumerate}
    \onslide\item<1-> \textbf{Definition:} Given two
sequences $X$ = $\langle$ $x_1$, $x_2$,...,$x_m$ $\rangle$ and $Z$  = $\langle$ $z_1$, $z_2$,...,$z_k$ $\rangle$, we say that $Z$ is a subsequence of $X$ if there is a strictly increasing sequence of $k$ indices $i_1$, $i_2$,...,$i_k$  $(1$ $\leq$ $i_1$ $<$ $i_2$ $<$ ... $<$ $i_k$ $\leq$ $m$) such that $Z$ = $\langle$ $x_{i_1}$, $x_{i_2}$,..., $x_{i_k}$ $\rangle$. 
    \onslide\item<2-> \textbf{Example:} Suppose $X$ = $\langle$LIFESUCKS$\rangle$. Then $\langle$LFS$\rangle$, $\langle$ISUCKS$\rangle$, $\langle$FES$\rangle$ etc. can be  subsequences of $X$. But $\langle$FIC$\rangle$, $\langle$ARA$\rangle$, $\langle$ERS$\rangle$ etc. are not subsequences of $X$. 
\end{enumerate}
\end{frame}


\begin{frame}{An Example}
\setbeamercovered{dynamic}
\begin{enumerate}
    \onslide\item<1-> \textbf{Input:} $s$ = $\langle$BDCB$\rangle$,  $t$ = $\langle$BACDB$\rangle$
    \onslide\item<2-> \textbf{Common Subsequences:} $\langle$B$\rangle$, $\langle$BD$\rangle$, $\langle$BCB$\rangle$ etc.
    \onslide\item<3-> \textbf{Longest Common Subsequence:} $\langle$BCB$\rangle$
\end{enumerate}
\end{frame}



\begin{frame}{Another Example}
\setbeamercovered{dynamic}
\begin{enumerate}
    \onslide\item<1-> \textbf{Input:} $s$ = $\langle$DABKC$\rangle$,  $t$ = $\langle$APBCK$\rangle$
    \onslide\item<2-> \textbf{Common Subsequences:} $\langle$A$\rangle$, $\langle$AB$\rangle$, $\langle$ABK$\rangle$, $\langle$ABC$\rangle$ etc.
    \onslide\item<3-> \textbf{Longest Common Subsequence:} $\langle$ABK$\rangle$, $\langle$ABC$\rangle$
\end{enumerate}
\end{frame}


\begin{frame}{Longest Common Subsequence Problem}
\begin{block}{Back to the Problem}
How do we find the longest common subsequence of two strings?
\end{block}
\end{frame}

\begin{frame}{Brute Force Algorithm}
\setbeamercovered{dynamic}
\begin{enumerate}
    \onslide\item<1-> Enumerate all subsequences of $s$, and check if they are subsequences of $t$.
    %\onslide\item<2-> Common Subsequence:
    % \onslide\item<3-> Longest Common Subsequence:
\end{enumerate}
\end{frame}


\begin{frame}{Brute Force Algorithm Complexity}
\setbeamercovered{dynamic}
\begin{enumerate}
    \onslide\item<1-> Suppose $s$ has length $n$ and $t$ has length $m$.
    \onslide\item<2-> Then there are $2^n$ subsequences of $s$. Checking each of them if they are subsequence of $t$ can be done greedily in $O(m)$. 
    \onslide\item<3-> So overall complexity is $O(m*2^n)$.
    \onslide\item<3-> Unfortunately even the fastest computer today can't complete calculations in thousands of years if $n$, $m$ $\geq$ 100. So we need faster algorithm. Dynamic programming in the rescue!
\end{enumerate}
\end{frame}



\begin{frame}{A Far Better Approach}
\setbeamercovered{dynamic}
\begin{enumerate}
    \onslide\item<1-> s: t:
    % \onslide\item<2-> Common Subsequence:
    % \onslide\item<3-> Longest Common Subsequence:
\end{enumerate}
\end{frame}



\begin{frame}{An Example Using Dynamic Programming Algorithm}
\setbeamercovered{dynamic}
% \caption{"ABCD"}
% \begin{enumerate}
%     \onslide\item<1-> s = ADAPT t=DBPT
%     % \onslide\item<2-> 
%     % \onslide\item<3-> Longest Common Subsequence:
% \end{enumerate}
% \bottom X = ADAPT Y = DBPT
\centering
$
\begin{array}{ | l | l | l | l | l | l | l | }
\hline
	 & j & 0 & 1 & 2 & 3 & 4 \\ \hline
	i &  & y & D & B & P & T \\ \hline
	0 & x & 0 & 0 & 0 & 0 & 0 \\ \hline
	1 & A & 0 & 0 & 0 & 0 & 0 \\ \hline
	2 & D & 0 & 1 & 1 & 1 & 1 \\ \hline
	3 & A & 0 & 1 & 1 & 1 & 1 \\ \hline
	4 & P & 0 & 1 & 1 & 2 & 2 \\ \hline
	5 & T & 0 & 1 & 1 & 2 & 3 \\ \hline
\end{array}
$
\end{frame}



\begin{frame}{Complexity}
\setbeamercovered{dynamic}
\begin{enumerate}
    \onslide\item<1-> s: t:
    % \onslide\item<2-> Common Subsequence:
    % \onslide\item<3-> Longest Common Subsequence:
\end{enumerate}
\end{frame}


% \begin{frame}{Reconstructing the Longest Common Subsequence}
% \setbeamercovered{dynamic}
\subfile{ConstructionTable.tex}

\subfile{SolnConstruction.tex}
% \begin{enumerate}
%     \onslide\item<1-> s: t:
%     % \onslide\item<2-> Common Subsequence:
%     % \onslide\item<3-> Longest Common Subsequence:
% \end{enumerate}
% \end{frame}




\begin{frame}{Summary}
  \begin{itemize}
  \item
    The \alert{first main message} of your talk in one or two lines.
  \item
    The \alert{second main message} of your talk in one or two lines.
  \item
    Perhaps a \alert{third message}, but not more than that.
  \end{itemize}
  
  \begin{itemize}
  \item
    Outlook
    \begin{itemize}
    \item
      Something you haven't solved.
    \item
      Something else you haven't solved.
    \end{itemize}
  \end{itemize}
\end{frame}



% All of the following is optional and typically not needed. 
\appendix
\section<presentation>*{\appendixname}
\subsection<presentation>*{For Further Reading}

\begin{frame}[allowframebreaks]
  \frametitle<presentation>{For Further Reading}
    
  \begin{thebibliography}{10}
    
  \beamertemplatebookbibitems
  % Start with overview books.

  \bibitem{Author1990}
    A.~Author.
    \newblock {\em Handbook of Everything}.
    \newblock Some Press, 1990.
 
    
  \beamertemplatearticlebibitems
  % Followed by interesting articles. Keep the list short. 

  \bibitem{Someone2000}
    S.~Someone.
    \newblock On this and that.
    \newblock {\em Journal of This and That}, 2(1):50--100,
    2000.
  \end{thebibliography}
\end{frame}

\end{document}


