\documentclass{report}
\usepackage[utf8]{inputenc}
\usepackage{array}
\usepackage{color}
\usepackage{hyperref} 

\newcommand{\RNum}[1]{\uppercase\expandafter{\romannumeral #1\relax}}
\newcolumntype{L}[1]{>{\raggedright\let\newline\\\arraybackslash\hspace{0pt}}m{#1}}
\newcolumntype{C}[1]{>{\centering\let\newline\\\arraybackslash\hspace{0pt}}m{#1}}
\newcolumntype{R}[1]{>{\raggedleft\let\newline\\\arraybackslash\hspace{0pt}}m{#1}}

\title{
\line(1,0){300}
\endgraf\bigskip
\Huge
\bfseries{A Report on}
\endgraf\bigskip
\emph{The Elements of Style}
\newline
\line(1,0){300}
\bigskip
\bigskip
}

\author{\Large{Anik Sarker}}

\date{
\Large{Student ID : 1405001}
\endgraf\bigskip
\Large{\today}
}

\newcommand{\SingleRowSingleCol}[1]{
    \begin{center}
    \begin{tabular}{|c|}
     \hline
     #1\\\hline
    \end{tabular}
    \end{center}
}

\newcommand{\DoubleRowSingleCol}[2]{
    \begin{center}
    \begin{tabular}{|c|}
     \hline
     #1\\\hline
     #2\\\hline
    \end{tabular}
    \end{center}
}

\newcommand{\TripleRowSingleCol}[3]{
    \begin{center}
    \begin{tabular}{|c|}
     \hline
     #1\\\hline
     #2\\\hline
     #3\\\hline
    \end{tabular}
    \end{center}
}

\newcommand{\DoubleRowDoubleCol}[4]{
    \begin{center}
    \begin{tabular}{|c|c|}
     \hline
     #1 & #2\\\hline
     #3 & #4\\\hline
    \end{tabular}
    \end{center}
}

\begin{document}

\maketitle
\renewcommand{\familydefault}{\sfdefault}

\tableofcontents

\chapter{INTRODUCTION}
This report briefly illustrates the basic features of the book \textit{Elements of Style}. The mentioned book basically discusses on the rules of usage and principles of writing composition. This report analyzes the rules of writing which are described in the book and briefly presents short notes on each of them. 

\bigskip

Although both the book and the report covers only a small portion of the field of English style, but this book is helpful to provide a guideline to inexperienced new writers.

\bigskip

The main objective of this report is to demonstrate all the basic points of the discussed book in short, provide examples for them, and provide exceptions to each rule if exists. 

\newpage


\chapter{ELEMENTARY RULES OF USAGE}{\label{chap:lab}}
\section*{1. Form the possessive singular of nouns with ’s}
\addcontentsline{toc}{section}{1. Form the possessive singular of nouns with ’s}

According to the usage of the United States Government Printing Office and of the Oxford
University Press, the possessive singular of nouns are formed with ’s irrespective of the final consonant.\\

\textbf{Example:}
\DoubleRowSingleCol{Sagor’s book}{Karim’s watch}

{\Large\textbf{Exception:}}
The possessives of all proper names ending in -s are formed by adding a single apostrophe after the last consonant.

\endgraf\bigskip

\textbf{Example:}
\DoubleRowSingleCol{François’ efforts}{Euripides’ tragedies}
~~~~The pronominal possessives hers, its, theirs, yours, and oneself have no apostrophe.

\newpage

\section*{2. In a series of three or more terms with a single conjunction, use a comma after each term except the last}
\addcontentsline{toc}{section}{2. In a series of three or more terms with a single conjunction, use a comma after each term except the last}


According to the usage of the Government Printing Office and the Oxford University
Press, a comma is used after each item until the last one in a list of three or more items
with a single 'and'/'or'.\\

\textbf{Example:}
\DoubleRowSingleCol{Bangladesh, Australia, and Zimbabwe}
{An Italian painter, sculptor, and architect}
\endgraf\bigskip


{\Large\textbf{Exception:}}
Business farm names are an exception to this rule. They omit the last comma.
\endgraf\bigskip

\SingleRowSingleCol{Rahim,Karim and Companies}
\endgraf\bigskip

\section*{3. Enclose parenthetic expressions between commas}
\addcontentsline{toc}{section}{3. Enclose parenthetic expressions between commas}

Non-restrictive relative clauses also follow this rule

\endgraf\bigskip

\textbf{Example:}
\DoubleRowSingleCol
{My father, a retired government officer, speaks Bangla, French, and English}
{Charlie Smith, who used to cook for us,has just opened his own restaurant}
\endgraf\bigskip


{\Large\textbf{Exception:}}
If application of this rule causes interruption to the flow of the sentence, both the commas can be omitted. We must never omit one comma and leave the
other. 
\endgraf\bigskip

\textbf{Example:}
\SingleRowSingleCol{A retired government officer my father speaks Bangla, French, and English}

\newpage

Restrictive relative clauses also do not follow this rule.
\SingleRowSingleCol{He who must not be named will be defeated by Harry Potter}

when a conjunction precedes a parenthetic expression, the first comma should be placed before the conjunction, not after it.
\SingleRowSingleCol{I went outside,but unsure of the weather, did not take the umbrella}
\endgraf\bigskip

\section*{4. Place a comma before \textit{and} or \textit{but} introducing an independent clause}

\addcontentsline{toc}{section}
{4. Place a comma before \textit{and} or \textit{but} introducing an independent clause}

In a compound sentence or compound predicate, if each clause can stand alone, a comma should always be used before the coordinating conjunction, but if only one clause can stand alone, a comma should never be used.

\textbf{Example:}
\DoubleRowSingleCol
{He tried everything but succeeded at nothing.}
{He tried everything, but he succeeded at nothing.}

If a dependent clause, or an introductory phrase requiring to be set off by a comma,
precedes the second independent clause, no comma is needed after the conjunction.
\endgraf\bigskip

\SingleRowSingleCol
{He tried everything, but since he was not powerful, he could not help.}
\endgraf\bigskip

\section*{5. Do not join independent clauses by a comma}
\addcontentsline{toc}{section}{5. Do not join independent clauses by a comma}
When a compound sentence is formed by concatenation of two or more independent clauses not joined by a conjunction, semicolon or period is used between them.

\textbf{Example:}
\DoubleRowSingleCol
{I bought a book recently; it is based on our war of liberation.}
{I bought a book recently. It is based on our war of liberation.}

If the clauses are joined by anything except conjunctions (such as adverbs like accordingly, besides,so, then, therefore, or thus), this rule holds.
\SingleRowSingleCol{I read the book already; so I did not buy it again.}


\newpage

{\Large\textbf{Exception:}}
If the clauses are very short, and are identical in form, a comma can be used.
\endgraf\bigskip

\textbf{Example:}
\SingleRowSingleCol{We tried, we failed.}
\endgraf\bigskip

\section*{6. Do not break sentences in two}
\addcontentsline{toc}{section}{6. Do not break sentences in two}
We should never use periods instead of commas. 
\DoubleRowSingleCol
{Wrong : I bought a fish from the market. When I was going to my office.}
{Right : I bought a fish from the market, when I was going to my office.}

{\Large\textbf{Exception:}}
An emphatic word or expression can be allowed to serve the purpose of a sentence.
\endgraf\bigskip

\textbf{Example:}
\SingleRowSingleCol{Run for your life. Go.}
\endgraf\bigskip

\section*{7. A participial phrase at the beginning of a sentence must refer to the grammatical subject}
\addcontentsline{toc}{section}{7. A participial phrase at the beginning of a sentence must refer to the grammatical subject}
Violation of this rule often makes the sentence unreasonable or refers to a incorrect meaning.

\textbf{Example:}
\DoubleRowSingleCol
{Wrong : While barking on the street, I threw a stone at the dog.}
{Right : I threw a stone at the dog when it was barking on the street.}
\bigskip

\DoubleRowSingleCol
{Wrong : Being difficult, I completed the task carefully.}
{Right : Being difficult, the task was completed by me carefully.}
\bigskip

\DoubleRowSingleCol
{Wrong : Winner of the competition, the chief guest awarded Russell.}
{Right : Winner of the competition, Russell was awarded by the chief guest.}
\newpage

\section*{8. Divide words at line-ends, in accordance with their formation and
pronunciation}
\addcontentsline{toc}{section}{8. Divide words at line-ends, in accordance with their formation and pronunciation}

If there is room at the end of a line for one or more syllables of a word, but not for
the whole word, we should divide the word. But if this involves cutting off only a single letter, or cutting off only two letters of a long word, this should be avoided. Mostly followed principles are:\\

~~~~A. We should divide the word according to its formation.
\TripleRowSingleCol
{wonder-ful (not wond-erful)}
{state-ment (not stat-ement)}
{terror-ism (not terro-rism)}
\bigskip

~~~~B. We should Divide on the vowel
\DoubleRowDoubleCol{secre-tary}{possi-ble}{exce-llent}{erro-neous}
\bigskip

~~~~C. We should divide between double letters.
\DoubleRowDoubleCol{hor-rible}{can-non}{mir-ror}{map-ping}
\newpage


\chapter{ELEMENTARY PRINCIPLES OF COMPOSITION}
In previous section {\ref{chap:lab}}, we discussed elementary rules of usage. Here, we discuss elementary principles of usage.



\section*{9. Make the paragraph the unit of composition: one paragraph to each topic}
\addcontentsline{toc}{section}{9. Make the paragraph the unit of composition: one paragraph per topic}

If the subject of writing extends to a small range, then the topic does not need any subdivision. Otherwise, the subject needs to be divided into ingredient topics and each of the topics should be subject of distinct paragraph.
\endgraf\bigskip
Single sentences should not be written or printed as a paragraph except a sentences of transition. The extent of subdivision will depend on the length of the composition.\\

\textbf{\Large{Example:}}
\endgraf\bigskip
\textbf{Paragraphs of a lab report}
\begin{enumerate}
  \item Objective
  \item Equipments
  \item Procedure
  \item Calculation
  \item Result
  \item Observations
\end{enumerate}

\bigskip

Any such writing must totally evaluate the nature of the writing and depth of the understanding of the writer.

\section*{10. As a rule, begin each paragraph with a topic sentence; end it in conformity with the beginning.}
\addcontentsline{toc}{section}{10. As a rule, begin each paragraph with a topic sentence; end it in conformity with the beginning.}

Each paragraph should begin with such a sentence which expresses the whole thought of the paragraph. The succeeding sentences explain or establish or develop the statement
made in the topic sentence. the final sentence either emphasizes the thought of the topic sentence or states some important consequence.\\

\textbf{Example:}

\begin{center}
\begin{tabular}{|C{7cm}|C{5cm}|}
 \hline
 1. Our thoughts and judgements are nothing but reflection of our belief.
 & 1. Topic sentence\\\hline
 2. Whatever happens around us, we judge it based on what we believe ourselves.
 & 2. The meaning of the topic sentence made clearer.\\\hline
 3. When we judge any event or analyze different points of the event, we always try to explain the course of events from our point of view.
 & 3. Logic on behalf of the topic sentence.\\\hline
 4. We shape the details of the events in a way which favours our belief and thought.
 & 4. One more reason on behalf of the topic sentence.\\\hline
 5. As a result, we give significance to the matters which favours us and ignore matters which goes against us.
 & 5. Effect of the event described.\\\hline
 6. To tell the truth, we see what we want to see, we don't see what we don't want to see even if we see.
 & 6. Conclusion reflecting the meaning of the topic sentence.\\\hline
\end{tabular}
\end{center}

\endgraf\bigskip

\textbf{\Large{Exception:}}
The brief paragraphs of animated narrative, however, are often without even this semblance
of a topic sentence. The break between them serves the purpose of a rhetorical pause, throwing into prominence some detail of the action.

\section*{11. Use the active voice}
\addcontentsline{toc}{section}{11. Use the active voice}

Active voice is more preferable to use than passive voice is. Active voice is usually more direct, vigorous, and alive.\\

\textbf{Example:}

\begin{center}
\begin{tabular}{|C{6cm}|C{6cm}|}
 \hline
 \textcolor{red}{Wrong} & \textcolor{green}{Right}\\\hline
 1. Rice will be eaten by me
 & 1. I will eat rice.\\\hline
 2. The project is being completed
 & 2. We are completing the project.\\\hline
 3. The bell was rung.
 & 3. I rang the bell.\\\hline
\end{tabular}
\end{center}

As a rule, one passive should never be made dependent directly upon another passive.
\begin{center}
\begin{tabular}{|C{5.5cm}|C{5.5cm}|}
 \hline
 \textcolor{red}{Wrong} & \textcolor{green}{Right}\\\hline
 1. I was not granted to be awarded.
 & 1. My award was not granted.\\\hline
\end{tabular}
\end{center}

\endgraf\bigskip

\textbf{\Large{Exception:}}
The need of making a particular word the subject of the sentence often determines which voice needs to be used. So, sometimes passive voice is preferred.\\

\textbf{Example:}

\begin{center}
\begin{tabular}{|C{5.5cm}|C{5.5cm}|}
 \hline
 \textcolor{red}{Wrong} & \textcolor{green}{Right}\\\hline
 1. The people of the community awarded the scientist for his invention.
 & 1. The scientist was awarded for his invention.\\\hline
\end{tabular}
\end{center}

\endgraf\bigskip

\section*{12. Put statements in positive form}
\addcontentsline{toc}{section}{12. Put statements in positive form}

Instead of using forced, unpopular, and colorless language, we should make definite assertions. It is better to express a negative in positive form.\\ 

\textbf{Example:}

\begin{center}
\begin{tabular}{|c|c|}
 \hline
 \textcolor{red}{Wrong} & \textcolor{green}{Right}\\\hline
 1. He was not very regular.
 & 1. He was very irregular.\\\hline
 2. He could not remember what happened.
 & 2. He forgot what happened.\\\hline 
 3. He did not pay attention to my words.
 & 3. He ignored my words.\\\hline 
\end{tabular}
\end{center}

\textbf{\Large{Exception:}}
Sometimes negative and positive words are used together for the necessity of the sentence. This kind of antithesis express special strong meaning.\\

\textbf{Example:}

\DoubleRowSingleCol
{Not that Karim did not play well, but Rahim played better.}
{Not only you, but also Sagor will come today.}

\newpage

\section*{13. Omit needless words}
\addcontentsline{toc}{section}{13. Omit needless words}

A paragraph must be as concise as possible. There should be no extra word in a sentence and no extra sentence in a paragraph.\\

\textbf{Example:}
\begin{center}
\begin{tabular}{|c|c|}
 \hline
 \textcolor{red}{Wrong} & \textcolor{green}{Right}\\\hline
 1. The house of Peter
 & 1. Peter's house\\\hline
 2. in a wonderful way
 & 2. wonderfully\\\hline
 3. The fact that I had achieved
 & 3. My achievement\\\hline
 4. My friend, who is today's chief guest
 & 4. My friend is today's chief guest \\\hline
\end{tabular}
\end{center}

If a series of sentences are sequential in nature, then they can be combined into a single complex sentence.

\endgraf\bigskip

\section*{14. Avoid a succession of loose sentences}
\addcontentsline{toc}{section}{14. Avoid a succession of loose sentences}

Loose sentences consisting of two co-ordinate clauses, where the second is introduced by a conjunction or relative, should not be repeated in two successive sentences. These kind of succession makes the text monotonous and tedious.\\

In such cases, the text should be recast to remove the monotony, replacing successive loose sentences by simple sentences, by sentences of two clauses joined by a semicolon, by periodic sentences of two clauses or by sentences of three clauses.\\

\textbf{Example:}
\begin{center}
\begin{tabular}{|C{5.5cm}|C{5.5cm}|}
 \hline
 \textcolor{red}{Wrong} & \textcolor{green}{Right}\\\hline
 I came home from office, and soon my doorbell rang. I quickly got up to open the door, but there was no one at the doorstep.
 & I came home from office; soon my doorbell rang. I quickly got up to open the door; there was no one at the doorstep.\\\hline
\end{tabular}
\end{center}

\newpage

\section*{15. Express co-ordinate ideas in similar form}
\addcontentsline{toc}{section}{15. Express co-ordinate ideas in similar form}

Expressions of similar content and function should be outwardly similar. The likeness of form enables the reader to recognize more readily the likeness of content and function.

\endgraf\bigskip

\textbf{Example:}
\begin{center}
\begin{tabular}{|C{5.5cm}|C{5.5cm}|}
 \hline
 \textcolor{red}{Wrong} & \textcolor{green}{Right}\\\hline
 1. Do or you die.
 & 1. Do or die.\\\hline

 2. Government of the people, by and for the people.
 & 2. Government of the people, by the people, for the people.\\\hline
 3. I will be playing football in the field, while classes will be being taken.
 & 3. I will be playing football in the field and the teacher will be taking classes.\\\hline
 4. Either he is the chairman or she.
 & 4. Either he is the chairman or she is.\\\hline
\end{tabular}
\end{center}

\endgraf\bigskip

\textbf{\Large{Exception:}}

\endgraf\bigskip

When a statement needs to be repeated in order to emphasize it, they may need to be of different form.

\section*{16. Keep related words together}
\addcontentsline{toc}{section}{16. Keep related words together}

Groups of words, that are related in thought, must be brought together and words which are not so related, should be kept apart.

\endgraf\bigskip

\textbf{Example:}

\endgraf\bigskip

\begin{enumerate}
  \item 
     The subject of a sentence and the principal verb should not be separated by a phrase or clause that can be transferred to the beginning.
     
    \begin{center}
    \begin{tabular}{|C{5cm}|C{5cm}|}
     \hline
     \textcolor{red}{Wrong} & \textcolor{green}{Right}\\\hline
     1. Mr. Rahman, when he founded this school, established a charity fund.
     & 1. When Mr. Rahman founded this school, he established a charity fund.\\\hline
     2. Our textbook, in the last few years, has gone through a lot of changes.
     & 2. In the last few years, our textbook, has gone through a lot of changes.\\\hline
    \end{tabular}
    \end{center}
    \newpage

  \item
     The relative pronoun should come, as a rule, immediately after its antecedent.
      
     \begin{center}
     \begin{tabular}{|C{5cm}|C{5cm}|}
     \hline
    \textcolor{red}{Wrong} & \textcolor{green}{Right}\\\hline
     1. There was a tone in his voice that expressed disagreement.
     & 1. In his voice was a tone that expressed disagreement.\\\hline
    \end{tabular}
    \end{center}
    \bigskip

  \item 
     To make sure that no ambiguity is caused, if the antecedent consists of a group of words, the relative comes at the end of the group.
      
    \begin{center}
    \begin{tabular}{|C{5cm}|C{5cm}|}
    \hline
     \textcolor{red}{Wrong} & \textcolor{green}{Right}\\\hline
     1. Bellatrix cursed Hermione fiercely, who was already wounded.
     & 1. Bellatrix, who was already wounded, cursed Hermione fiercely.\\\hline
    \end{tabular}
    \end{center}
    \bigskip

  \item 
     As a noun in apposition does not arise ambiguity, when placed between antecedent and relative, it is allowed.
    \SingleRowSingleCol{Bellatrix, the fierce death-eater, who was already wounded}
    \bigskip
      
  \item
     Modifiers should be placed right after the word they modify.
    \begin{center}
    \begin{tabular}{|C{5cm}|C{5cm}|}
    \hline
    \textcolor{red}{Wrong} & \textcolor{green}{Right}\\\hline
     1. He was again accused of theft.
     & 1. He was accused of theft again.\\\hline
     2. All of the apples were not rotten.
     & 2. Not all of the apples were rotten.\\\hline
    \end{tabular}
    \end{center}    
\end{enumerate}

\newpage

\section*{17. In summaries, keep to one tense}
\addcontentsline{toc}{section}{17. In summaries, keep to one tense}

We should always try to use one tense in our writing. While summarizing the action of a drama, present tense should be always used. While discussing on a poem, story or novel, present tense is preferable, but past tense can also be used. If the summary is in the present tense, antecedent action should be expressed by the perfect; if in the past, by the past perfect. 

\endgraf\bigskip

\textbf{Example:}
\begin{center}
\begin{tabular}{|C{11cm}|}
 \hline
 When Bruce was looking through the window, he saw someone coming towards his cave.
 He remembered that he had asked Robert to come and meet him the other night. Bruce quickly dressed up and stood at the door to welcome Robert. Soon Robert arrived at the doorstep.\\\hline
\end{tabular}
\end{center}    

\endgraf\bigskip

\section*{18. Place the emphatic words of a sentence at the end}
\addcontentsline{toc}{section}{18. Place the emphatic words of a sentence at the end}

The main keywords or clause of a sentence, which we want to present most prominently, should be placed at the end of the sentence.

\endgraf\bigskip

\textbf{Example:}
\begin{center}
\begin{tabular}{|C{5.5cm}|C{5.5cm}|}
 \hline
 \textcolor{red}{Wrong} & \textcolor{green}{Right}\\\hline
 1. Farmers of our country are not educated, though they are industrious.
 & 1. Farmers of our country are industrious, but they are not educated.\\\hline
 2. Marconi gave birth to wireless technology, by inventing radio.
 & 2. By inventing radio. Marconi gave birth to wireless technology.\\\hline
\end{tabular}
\end{center}

\endgraf\bigskip

\textbf{\Large{Exception:}}

\endgraf\bigskip

Sometimes, beginning can also be the prominent position. Any element in the
sentence, other than the subject, becomes emphatic when placed first.

\DoubleRowSingleCol
{Humiliation or Death - you have only one choice.}
{Gorgeous and spacious, this is our new apartment.}

\newpage

\chapter{A FEW MATTERS OF FORM}
\begin{itemize}
  \item
   \textbf{\Large{Headings:}}\\
    After the title or heading of a manuscript, we should leave a blank line, or its equivalent in space. We should begin on the first line on succeeding pages.
    \bigskip
    
  \item
   \textbf{\Large{Numerals:}}\\
    We should not spell out dates or other serial numbers. Rather we should write them in figures or in Roman notations.
    \DoubleRowDoubleCol{Theorem 5}{September 1, 1997}{Verse \RNum{27}}{2nd boy}
    \bigskip

  \item
   \textbf{\Large{Parentheses:}}\\
    We should punctuate a sentence containing an expression in parenthesis 
    outside of the marks of parenthesis, exactly as if the expression in parenthesis
    were absent. We should punctuate the expression within the prenthesis as if it stood by itself.
    \DoubleRowSingleCol
    {The general summoned his force (the 8th battalion) and left the war.}
    {He tried to save his country (I don't know the name), but he failed.}
    \bigskip

  \item
   \textbf{\Large{Quotations:}}\\
    We should introduce Formal quotations, cited as documentary evidence, by a colon and enclose them in quotation marks.
    \SingleRowSingleCol{The motto of the school is : "Truth shall prevail."}
    \newpage
    If quotations are in apposition or in the direct objects of verb, then they are preceded by a comma and enclosed in quotation marks.
    \SingleRowSingleCol{Kate asked, "Will you go ?"}

   \textbf{Exception:}\\
    Quotations of an entire line or more, of verse, quotations introduced by 'that' and quotations containing proverbial expressions and familiar phrases of literary origin does not require any parentheses.
    
    \begin{center}
    \begin{tabular}{|C{6cm}|}
     \hline
     Charles Dickens wrote :
     \\Keep me through this night of peril
     \\Underneath its boundless shade;\\\hline
     I said that I wil change the system.\\\hline
    \end{tabular}
    \end{center}
    \bigskip

  \item
   \textbf{\Large{References:}}\\
    We should abbreviate titles that occur frequently and give the full forms in an alphabetical list at the end, in scholarly work requiring exact references. We should give the references in parenthesis or in footnotes, not in the body of the sentence. We should avoid words like act, scene, line, book, volume, page, except etc.
    
    \begin{center}
    \begin{tabular}{|C{5.5cm}|C{5.5cm}|}
     \hline
     The chapter of a big hero ended with the defeat
     of Napoleon in the battle of Waterloo (ref:1).
     & (Footnote) 1. The battle of Waterloo : fought on Sunday, 18 June 1815,
     near Waterloo in present-day Belgium.\\\hline
    \end{tabular}
    \end{center}
    \bigskip
    
   \item
    \textbf{\Large{Titles:}}\\
    For the titles of literary works, We should prefer italics with capitalized initials.
    Also, Roman with capitalized initials and with or without quotation marks can be used.

    \begin{center}
    \begin{tabular}{|C{6cm}|}
     \hline
     \textit{Harry Potter and the Prisoner of Azkaban, Around the World in Eighty Days,
     The Passage, The Sun Also Rises, The Wind in the Willows.}\\\hline
    \end{tabular}
    \end{center}
    
    \newpage    
\end{itemize}


\chapter{WORDS AND EXPRESSIONS COMMONLY MISUSED}

Some words in English language are commonly mistaken frequently. These mistakes arises when proper care is not taken while writing. the proper correction for these mistakes is not the replacement of one word or set of words by another, but the replacement of vague generality by definite statement.
\bigskip

\begin{itemize}
    \item
    \textbf{\Large{Allude:}}
    
    \textit{Allude} means \textit{to suggest or call attention to indirectly}, not \textit{Elude} or \textit{to evade or escape from something in a cunning way}.
    \DoubleRowSingleCol
    {Correct : You allude to a book}
    {Correct: You elude a pursuer.}
    
    \bigskip
    
    \item
    \textbf{\Large{Clever:}}\\
    \textit{Clever} means \textit{ingenious} when applied to people, but when applied to horses, it means \textit{a good-natured one}.
    
    \DoubleRowSingleCol
    {Statement : Mr. Brown is a clever guy (means the person is cunning)}
    {Statement : The horse is very clever. (means the horse is obedient.)}
    \bigskip

    \item
    \textbf{\Large{Disinterested :}}\\
    \textit{Disinterested} means unbiased and does not mean \textit{uninterested}.
    
    \DoubleRowSingleCol
    {Correct : The dispute should be resolved by a disinterested judge.}
    {Correct : Why are you so uninterested in my story ?}
     
    \newpage
    
    \item
    \textbf{\Large{Farther:}}\\
    
    \textit{Farther} is commonly interchanged with \textit{further}. But they have different meaning. \textit{farther} is a word implying distance, while \textit{further} is a measurement of time or quantity.
    
    \DoubleRowSingleCol
    {Correct : You chase a ball farther than the other fellow.}
    {Correct: You pursue a subject further.}
        .
    \item
    \textbf{\Large{In terms of:}}\\
    
    \textit{In terms of} actually does not make much important meaning to the reader. It is better if it is omitted.
    \DoubleRowSingleCol
    {Correct : Let's talk in terms of science.}
    {Better: Let's talk scientifically.}
    \bigskip

    \item
    \textbf{\Large{Appraise :}}\\    
    \textit{Appraise} means to ascertain the value of and does not mean to \textit{apprise}.
    \DoubleRowSingleCol
    {Correct: I appraised the jewels.}
    {Correct : I apprised him of the situation.}
    \bigskip
    
    \item
    \textbf{\Large{Bemused :}}\\    
    \textit{Bemused} means bewildered and does not mean \textit{amused.}

    \DoubleRowSingleCol
    {Correct: The unnecessarily complex plot left me bemused.}
    {Correct : The silly comedy amused me.}
    \bigskip
    
    \item
    \textbf{\Large{Depreciate :}}\\  
    \textit{Depreciate} means to decrease in value and does not mean to \textit{deprecate} or to disparage.
    
    \DoubleRowSingleCol
    {Correct: My car has depreciated a lot over the years.}
    {Correct : She deprecated his efforts.}
\end{itemize}


\chapter{WORDS OFTEN MISSPELLED}
Some words in English are error-prone and complex in structure. A short list of them is presented below:

\bigskip
\fontsize{15}{18}

\begin{center}
\begin{tabular}{c c c}
 aggression & irresistible & principal\\
 apparently	& intelligence & questionnaire \\
 argument & jewelry & receipt\\
 beginning & kernel & rhyme\\
 bizarre & leisure & rhythm \\
 cemetery & liaison & sergeant \\
 colleague & lieutenant & supersede\\
 committee & maintenance & threshold\\
 dilemma & maneuver & tyranny\\
 ecstasy & medieval & vacuum\\
 Fahrenheit & minuscule & weather\\
 fluorescent & mischievous & weird\\
 foreign & misspell & xylophone\\
 foreseeable & occasionally & yacht\\
 gist & perseverance & zebra\\
 interrupt & personnel & zephyr\\
\end{tabular}
\end{center}
\bigskip

\chapter{CONCLUSION}{
    \textit{Elements of Style} is a unique book on rules for composition writing. it gives in brief space the principal requirements of plain English style and concentrates attention on the rules of usage and principles of composition most commonly violated.
    
    \bigskip
    
    Despite its brief and well-structured features, this book is not completely well-explained on all topics. A little more rich collection of examples can make this book more helpful for the readers and writers.
    
    \bigskip
    
    As a basic textbook for learning composition writing and overcoming common mistakes, this book is a very good choice.
    
    \bigskip
    
    For any query, please contact : 
    \href{http://aniksarkerbuet1997@gmail.com}{\textcolor{blue}{aniksarkerbuet1997@gmail.com}}
}

\end{document}
