\documentclass[a4paper,12pt,oneside]{article}
\usepackage{ulem}

\title{A1 CSE 300 Day 1}
\author{Md. Shaifur Rahman}

\begin{document}
\maketitle
\newpage
\tableofcontents
\newpage
\section{About BUET}
\subsection{Introduction}
\label{xyz}
\subsubsection{BUET}	
Bangladesh University of Engineering and Technology, abbreviated as \textbf{BUET}, is one of the most \textit{prestigious institutions} for higher studies in the country. About 5500 students are pursuing undergraduate and postgradute studies in engineering, architecture, planning and science in this institution. At present, BUET has sixteen teaching departments under five faculties and it has three institutes. Every year the intake of undergraduate students is around 900, while the intake of graduate students in Masters and PhD programs is around 1000. A total of about five hundred teachers are teaching in these departments and institutes. There are additional teaching posts like Dr. Rashid Professor, \uline{Professor Emeritus} and Supernumerary Professors.
\subsubsection{Now}
\noindent The BUET campus is in the heart of Dhaka – the capital city of Bangladesh. It has a compact campus with halls of residence within walking distances of the academic buildings. The physical expansion of the University over the last three decades has been impressive with construction of new academic buildings, auditorium complex, halls of residence, etc.
\subsection*{History}	
BUET is the oldest institution for the study of Engineering and Architecture in Bangladesh. The history of this institution dates back to the days of Dhaka Survey School which was established at Nalgola, in Old Dhaka in 1876 to train Surveyors for the then Government of Bengal of British India. As the years passed, the Survey School became the Ahsanullah School of Engineering offering three-year diploma courses in Civil, Electrical and Mechanical Engineering. In recognition of the generous financial contribution from the then Nawab of Dhaka, it was named after his father Khawja Ahsanullah. It moved to its present premises in 1912. In 1947, the School was upgraded to Ahsanullah Engineering College as a Faculty of Engineering under the University of Dhaka, offering four-year bachelor’s courses in Civil, Electrical, Mechanical, Chemical and Metallurgical Engineering. In order to create facilities for postgraduate studies and research, Ahsanullah Engineering College was upgraded to the status of a University in 1962 and was named East Pakistan University of Engineering and Technology. After the war of Liberation in 1971, Bangladesh became an independent state and the university was renamed as the Bangladesh University of Engineering and Technology.
	
Till today, \texttt{it has produced around 25,000 graduates in different branches of engineering and architecture}, and has established a good \LaTeX{} reputation all over the world for the quality of its graduates, many of whom have excelled in their profession in different parts of the globe. It was able to attract students from countries like India, Nepal, \tiny Iran, \Huge Jordan, \normalsize Malaysia, Sri Lanka, Pakistan and Palestine.
There are following departments in BUET- 
\begin{itemize}
\item CSE
	\begin{itemize}
		\item IAC
		\item CL
			\begin{itemize}
				\item PC
					\begin{itemize}
						\item Mainboard
						\item Processor
					\end{itemize}
				\item Projector
			\end{itemize}
		\item MML
	\end{itemize}
\item EEE 
\item IPE
\item ME
\end{itemize}

The labs are
\begin{description}
\item{IAC} Information Access Center
\item{CL} Computation Lab
\end{description}

\subsection{Conclusion}	
Both Undergraduate and Postgraduate studies and research are now \# \$ \^ among the primary functions of the University. Eleven departments under five faculties offer Bachelor Degrees, while most of the departments and institutes offer Masters Degrees and some of the departments have Ph.D. programs. In addition to its own research programs, the university undertakes research programs sponsored by outside organizations like European Union, UNO, Commonwealth, UGC, etc. The expertise of the University teachers and the laboratory (as discussed in section \ref{xyz} in page \pageref{xyz}) facilities of the University are also utilized to solve problems and to provide up-to-date engineering and technological knowledge to the various organizations of the country.
\end{document}
