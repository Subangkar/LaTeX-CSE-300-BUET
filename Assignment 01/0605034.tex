\documentclass[12 pt]{article}
\author{Sheikh Al Amin Sunoyon\\ Student ID: 0605034\\ \LaTeX}
\title{{An article on}\\ \textbf{\Huge{THE GREAT PYRAMIDS OF GIZA}}}
\date{}
\usepackage[pdftex]{hyperref}
\usepackage[pdftex]{graphicx}
\begin{document}
\maketitle

\newpage
\tableofcontents
\newpage
\setlength{\baselineskip}
{1.5\baselineskip}
\section{Introduction}
The {\bfseries Great Pyramid of Giza} (also called the {\bfseries \itshape Pyramid of King Khufu' and Pyramid of Cheops)} is the oldest and largest of the three pyramids in the Giza Necropolis bordering what is now El Giza, Egypt, and is the only one of the Seven Wonders of the Ancient World that survives substantially intact. It is believed the pyramid was built as a tomb for Fourth dynasty Egyptian {\itshape Pharoah Khufu (Cheops in Greek)} and constructed over a 20 year period concluding around 2540 BC. The Great Pyramid was the tallest man-made structure in the world for over 3,800 years. Originally the Great Pyramid was covered by casing stones that formed a smooth outer surface, and what is seen today is the underlying core structure. Some of the casing stones that once covered the structure can still be seen around the base. There have been varying scientific and alternative theories regarding the Great Pyramid's construction techniques. Most accepted construction theories are based on the idea that it was built by moving huge stones from a quarry and dragging and lifting them into place.

There are three known chambers inside the Great Pyramid. The lowest chamber is cut into the bedrock upon which the pyramid was built and was unfinished. The so-called Queen's Chamber and King's Chamber are higher up within the pyramid structure. The Great Pyramid of Giza is the main part of a complex setting of buildings that included two mortuary temples in honor of Khufu{\itshape (one close to the pyramid and one near the  Nile)}, three smaller pyramids for Khufu's wives, an even smaller "satellite" pyramid, a raised causeway connecting the two temples, and small mastaba tombs surrounding the pyramid for nobles.

\newpage

\begin{center}
\includegraphics{giza1.jpg}
\newline The Giza Pyramids\newline
\end{center}

\section{History}
It is believed the pyramid was built as a tomb for Fourth dynasty Egyptian pharaoh Khufu and constructed over a 14 to 20 year period concluding around 2540 BC. {\itshape Khufu's vizier, Hemon, or Hemiunu}, is believed by some to be the architect of the Great Pyramid. It is thought that, at construction, the Great Pyramid was originally 280 Egyptian cubits tall, 146.6 meters, (480.97 feet, or about 40 stories) but with erosion and the loss of its pyramidion, its current height is 138.8 m (455 feet). Each base side was 440 royal cubits, 230.5 meters in length, (756.2 feet). A royal cubit measures 0.524 meters. The total mass of the pyramid is estimated at 5.9 million tonnes. The volume, including an internal hillock, is believed to be roughly 2,500,000 cubic meters. Based on these estimates, building this in 20 years would involve installing approximately 800 tonnes of stone every day. {\itshape The first precision measurements of the pyramid were done by Egyptologist Sir Flinders Petrie in 1880–82 and published as The Pyramids and Temples of Gizeh. Almost all reports are based on his measurements}. Many of the casing stones and interior chamber blocks of the great pyramid were fit together with extremely high precision. Based on measurements taken on the north eastern casing stones, the mean opening of the joints are only 0.5 millimeters wide (1/50th of an inch).

\section{Interior Design}
The Great Pyramid is the only pyramid known to contain both ascending and descending passages. There are three known chambers inside the Great Pyramid. These are arranged centrally, on the vertical axis of the pyramid. From the entrance, an 18 meter corridor leads down and splits in two directions. One way leads to the lowest and unfinished chamber. This chamber is cut into the bedrock upon which the pyramid was built. It is the largest of the three, but totally unfinished, only rough-cut into the rock. The other passage leads to the Grand Gallery (49 m x 3 m x 11 m), where it splits again. One tunnel leads to the Queen's Chamber, a misnomer, while the other winds to intersect with the descending corridor. The Grand Gallery itself features a corbel haloed design and several cut "sockets" spaced at regular intervals along the length of each side of its raised base with a "trench" running along its center length at floor level. What purpose these sockets served is unknown. An antechamber leads from the Grand Gallery to the King's Chamber.
\subsection{Entrance}
Today, tourists enter the Great Pyramid via a forced tunnel dug by the Caliph Al-Ma'mum and his men around 820 AD. The tunnel continues for approximately 30 meters and eventually meets up with the Descending Passage which at the time was found to have been blocked by a series of massive granite plugs.

\begin{center}
\includegraphics{entrance.jpg}
\newline Entrance\newline
\end{center} 

Unable to remove the blocks, the workmen tunneled around the plugs discovering the Ascending Passage which leads to the Grand Gallery and interior chambers only to find them empty. The original entrance, which was apparently unknown at the time, can be seen today several meters directly above the forced entry and would have also been blocked by the granite plugs.
\subsection{King's Chamber}
At the end of the lengthy series of entrance ways leading into the interior is the structure's main chamber, the King's Chamber. This granite room was originally 10 × 20 × 11.4 cubits, or about 5.235 m × 10.47 m × 5.974 m, comprising a double 10 × 10 cubit square floor, and a height equal to half the double square's diagonal. Some believed that the height was consistent with the geometric methods for determining the Golden Ratio φ (phi) as the height is approximately phi times the width minus ½, while phi can be derived from other dimensions of the pyramid, but evidence from Petrie’s surveys and later conclusions drawn by others shows that it was in fact the circular proportions that were deliberately incorporated into the internal and external designs of the Great Pyramid by its architects and builders, for symbolic reasons. The so called golden ratio phi simply exists in the proportions of the architecture as an inadvertent by-product of the inclusion of the circular proportions. The reason for the inadvertent inclusion is that phi, the golden ratio, has a naturally occurring mathematical relation to the circular ratio pi that is unrelated to the architecture or geometry, and which was unknown to the pyramid's builders. Petrie confirmed that the King’s Chamber was a triumph of Egyptian geometry, the ratio of its length to the circuit of the side wall being the same as the ratio of 1 to pi, and that the exterior of the pyramid had been built to the same proportions.
\subsection{Queen's Chamber}

The Queen's Chamber is the middle and the smallest, measuring approximately 5.74 by 5.23 meters, and 4.57 meters in height. The chamber is lined with fine limestone blocks and the pented roof is made of large limestone slabs.Its eastern wall has a large angular doorway or niche. Egyptologist Mark Lehner believes that the Queen's chamber was intended as a serdab, a structure found in several other Egyptian pyramids, and that the niche would have contained a statue of the interred. The Ancient Egyptians believed that the statue would serve as a "back up" vessel for the Ka of the Pharaoh, should the original mummified body be destroyed. 

\section{Mistery}
The positioning of the three pyramids of Giza is a bit surprising. They are not quite in a straight line, clustered around the largest one, or grouped in any kind of expected symmetrical way. The proposed explanation of most Egyptologists is that this had something to do with the terrain at Giza or it was simply the way the construction worked out.

In the early 1990s, Belgian engineer Robert Bauval noticed that the odd arrangement of the Giza pyramids is remarkably similar to that of the three stars of Orion's belt in the well-known constellation. This seemed to Bauval to be more than a coincidence, in light of the fact that the constellation Orion was sacred to the Egyptians. They believed it to be the home of the god Osiris and thought the shape of the constellation resembled him.

Among the many fascinating features of the Giza pyramids are the four airshafts in the north and south faces of the King's Chamber of the Great Pyramid, and the two in the Queen's Chamber beneath it. Bauval calculated that in 2500 BC, the southern vent would have pointed directly at Orion and the southern airshaft in the Queen's Chamber would have pointed at the star Sirius, which was sacred to Osiris' consort Isis.


Bauval theorized that the vent was intended to be a channel to direct the pharaoh's soul to Orion, where he would become a god. Many scientists have dismissed Bauval's ideas, yet they certainly remain intriguing and continue to generate a great deal of discusssion. You can read more about it in the links listed at the end of this article.

Another interesting observation is that the Great Pyramid is perfectly aligned to true north, south, east and west. This has led to speculation about an astrological meaning to its position. A number of theories have been advanced concerning occult meanings, secret codes or prophecies derived from the pyramid's dimensions.

\section{Tourists' Information} 
There are many hotels in Giza,Egypt. These are the renowned hotels.
\begin{enumerate}
\item{\bfseries Pyramids Park Resort Cairo, Giza: }

Nestled in 25 hectares (62 acres) of lush gardens teeming with peacocks and pink flamingos, guests could spend days exploring the landscaped terraces and colossal free-form swimming pool here. The spacious rooms with all amenities overlook either the pool or garden and the Pyramids are just a stone's throw away. This is an ideal resort for business execs and holidaymakers, and the management has gone out of its way to make it unusually child-friendly.
\item{\bfseries Le Meridien Pyramids, Giza:}

Just a short walk from the Pyramids, almost half of the rooms in this beautiful hotel have stunning views of the World's Seventh Wonder, while others overlook the cool blue swimming pool. Operating as a self-contained village, this full-amenity hotel is a great option for those with only a few days to spend and an agenda for relaxation. Rooms are decorated in soft tones with ample room to spread out, and the hotel is very children-friendly.

\item{\bfseries Hotel Les Trois Pyramides, Giza:}

The mock pharaonic statuary at the entrance can be a little much, but most tourists who stay here enjoy the glitzy Ramesside theme. The hotel is, after all, located just minutes from its famous namesake, and many of its rooms offer peeks of the ancient Seventh Wonder of the World. It is also situated in the heart of Cairo's nightclub scene, and the energetic pulse of clubs and cabarets beats late into the night here.
\end{enumerate}



\section{Conclusion}


Giza is the most important site on earth for many New Age followers, who are drawn by the pyramids' mysteries and ancient origins. Since 1990, private groups have been allowed into the Great Pyramid, and the majority of these have been seekers of the mystical aspects of the site. But even the most skeptical visitor cannot help but be awed by the great age, grand scale and harmonic mathematics of the pyramids of Giza.

\section{Quick Facts}
\begin{itemize}
\item{Name:} {\itshape The Great Pyramids of Giza}
\item{Location:} {\itshape Giza, Egypt}
\item{Category:} {\itshape Graves and Tombs; World Heritage Sites}
\item{Architecture:} {\itshape Egyptian}
\end{itemize}

\section{References}
\begin{itemize}

\item \href {http://en.wikipedia.org/wiki/Great_Pyramid_of_Giza}{Article from Wikipedia}

\item \href {http://www.sacred-destinations.com/egypt/giza-pyramids}{Article from sacred-destination}

\end{itemize}




\end{document}