\documentclass{article}
\usepackage[utf8]{inputenc}

\title{Pythagorean Theorem}
\author{1505015}
\date{\today}

% \usepackage{lipsum}
\usepackage{subfiles}
\usepackage[algoruled, linesnumbered]{algorithm2e}
\newtheorem{theorem}{Theorem}[section]
\usepackage{listings}
\usepackage{color}
\usepackage{tikz}
\usepackage{amssymb}
\usepackage{amsmath}
\usepackage{xcolor}

\definecolor{background_trngl}{RGB}{215,235,242}
\definecolor{sq_a}{RGB}{164,117,169}
\definecolor{sq_b}{RGB}{245,168,150}
\definecolor{sq_c}{RGB}{29,181,220}
\definecolor{text_trngl}{RGB}{127,37,120}
\definecolor{circle}{RGB}{16,45,105}



\begin{document}

\maketitle
\section{Introduction}
In this document, we present the very famous theorem in mathematics:  \textit{Pythagorean
theorem}, which is stated as follows.

\subfile{PythThm.tex}
\subfile{PythTriangle.tex}
\newpage
\subfile{PythCircle.tex}


\section{Trigonometric Forms}
Lots of other forms of the same theorem exist. The most useful, perhaps, are
expressed in trigonometric terms, as follows:

\begin{equation}
    sin^2\theta + cos^2\theta = 1
\label{eqn:eqn1}
\end{equation}
\begin{equation}
    sec^2\theta - tan^2\theta = 1
\label{eqn:eqn2}
\end{equation}
\begin{equation}
    cosec^2\theta - cot^2\theta = 1
\label{eqn:eqn3}
\end{equation}

\subsection{Representing the First}
Taking \ref{eqn:eqn1}, we can show them as shown in Figure\ref{fig:fig2}. When we take a point at
unit distance from the origin, the y and x co-ordinates become sin$\theta$ and cos$\theta$
respectively. Therefore, sum of the squares of the two becomes equal to the
square of the unit distance, which of course, is 1.

\end{document}
