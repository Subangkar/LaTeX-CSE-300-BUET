\begin{figure}[h]
    \centering
    \begin{tikzpicture}[scale =0.5]
        
         \path [fill=background] (-7,7) -- (-7,-7) -- (7,-7) -- (7,7) -- (-7,7);
     
        
        
        \path [fill=a]  (1,0) -- (1,-2) -- (-1,-2) -- (-1,0) -- (1,0); % a fill
        \path [fill=b]  (1,0) -- (4.5,0) -- (4.5,4) -- (1,4) -- (1,0); %b fill
        \path [fill=c] ( 1,4) -- (-3,6) -- (-5,2) -- (-1,0) -- (1,4); %c fill        
        \draw [orange,line width=1pt] (1,0) -- (1,4) --  (-1,0) -- (1,0); % triangle
        

        \path [fill=a] (0,-3.5) -- (0,-5.5) --(2,-5.5) -- (2,-3.5) -- (0,-3.5); %a^2 fill
        \path [fill=b] (3,-3) -- (3,-6.25) -- (6.5,-6.25) -- (6.5,-3) --(3,-3); % b^2 fill
        \path [fill=c] (-6,-6.5) -- (-6,-2.5) -- (-1.5,-2.5) -- (-1.5,-6.5) -- (-6,-6.5) ; %c^2 fill

        % \draw (0,-3.5) -- (0,-5.5) --(2,-5.5) -- (2,-3.5) -- (0,-3.5); %a^2
        % \draw (3,-3) -- (3,-6.25) -- (6.5,-6.25) -- (6.5,-3) --(3,-3); % b^2
        % \draw (-6,-6.5) -- (-6,-2.5) -- (-1.5,-2.5) -- (-1.5,-6.5) -- (-6,-6.5) ; %c^2
        
        \draw [blue] (.6,0) -- (.6,.5) -- (1,.5);
        
        % \draw  (1,0) -- (1,-2) -- (-1,-2) -- (-1,0);    %a
        % \draw (1,4) -- (-3,6) -- (-5,2) -- (-1,0)   ;   %c
        % \draw (1,0) -- (4.5,0) -- (4.5,4) -- (1,4)  ;   %b
        
        


        
        
       
        \node [above,scale=1.5] at (-.75,-5) {\color{operator}{$=$}};
        \node [above,scale=1.5] at (2.5,-5) {\color{operator}{$+$}};
        
        \node [below,scale=1.5] at (0,0) {a} ;
        \node [right,scale=1.5] at (1,2) {b};
        \node [left,scale=1.5] at (0,2)  {c};
        
        \node [below,scale=1.75] at (-3.75,-3.5) {{\color{operator}{c$^2$}}};
        \node [below,scale=1.5] at (1,-3.75) {a$^2$};
        \node [below,scale=1.5] at (4.75,-3.75) {b$^2$};
        
       
    
    \end{tikzpicture}
    \caption{Visual representation of the famous Pythagorean theorem.}
    \label{fig:fig1}
\end{figure}